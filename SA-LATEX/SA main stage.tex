\documentclass[a4paper,14pt]{extarticle} % the default << article >> class is limited to 12pt, but you can go up to 14, 17 or 20 points if you use the << extarticle >> class
\usepackage{cmap} % make LaTeX PDF output copy-and-pasteable
\usepackage[T2A]{fontenc}
\usepackage[utf8]{inputenc}
\usepackage[english,ukrainian]{babel}

\usepackage{amssymb, amsfonts, mathtools, amsmath, enumerate}
\usepackage{indentfirst} % set an additional space before a paragraph at the begining of a new section
\usepackage{setspace}
\usepackage{textcomp}

\usepackage{dsfont} % indicator symbol
\usepackage{leftidx} % this package enables left subscripts and superscripts in math mode

\usepackage{import} % for adding a file by path https://tex.stackexchange.com/questions/246/when-should-i-use-input-vs-include

\usepackage{geometry} 
\geometry{left=1.25cm}
\geometry{right=1.25cm}
\geometry{top=1cm}
\geometry{bottom=2cm}

\setlength{\arrayrulewidth}{0.3mm} % this sets the thickness of the borders of the table
\setlength{\tabcolsep}{12pt} % the space between the text and the left/right border of its containing cell is set to 18pt with this command
\renewcommand{\arraystretch}{1.5} % the height of each row is set to 1.5 relative to its default height

\usepackage[table,xcdraw,dvipsnames]{xcolor}
\usepackage{color}
% 1) tutorial about xcolor:  https://www.overleaf.com/learn/latex/Using_colours_in_LaTeX
% 2) huge tutorial about xcolor: https://latex-tutorial.com/color-latex/ 
% 3) RGB calculator: https://www.w3schools.com/colors/colors_rgb.asp

\usepackage{hyperref}
\definecolor{linkcolor}{HTML}{0000FF}
\definecolor{urlcolor}{HTML}{0000FF} 
\hypersetup{pdfstartview=FitH, unicode=true, linkcolor=linkcolor, urlcolor=urlcolor, colorlinks=true}

\usepackage{graphicx}
\usepackage{wrapfig}
\usepackage{float}

\parskip=1mm % space between paragraphs

\usepackage{listingsutf8} % origin: \usepackage{listings}

\lstset{
    frame=single,
    language=Python,
    aboveskip=3mm,
    belowskip=3mm,
    columns=flexible,
    basicstyle={\small\ttfamily},
    numbers=left,
    numberstyle=\tiny\color{gray},
    commentstyle=\color{OliveGreen},
    stringstyle=\color{Mahogany},
    morestring=[b]''',
    showstringspaces=false,
    keywordstyle=\bfseries\color{blue},
    emph={[1]import, as, for, while, return}, emphstyle={[1]\bfseries\color{magenta}},
    emph={[2]range}, emphstyle={[2]\bfseries\color{brown}},
    breaklines=true,
    breakatwhitespace=true,
    tabsize=4,
    extendedchars=false, % to use ukrainian text in a code
    inputencoding=utf8 % to use ukrainian text in a code
}

\begin{document}

\import{Title/}{title}

\tableofcontents

\newpage
\import{LaTeX conversion stage/}{conversion_stage}

\newpage
\import{LaTeX UKR/}{ZNO_ukr}

\newpage
\import{LaTeX MATH/}{ZNO_math}

\import{LaTeX ENG/}{ZNO_eng}

\section*{Загальний висновок}
\addcontentsline{toc}{section}{Загальний висновок}

Метою статистичного аналізу було з'ясувати, чи є значущою відмінність між балами ЗНО для чоловіків 
та жінок. При цьому розглядалися результати трьох різних предметів: української мови, математики та 
англійської мови.

Перш за все вдалося встановити, що на рівні значущості $\alpha=0.01$ лише для української мови наявна 
залежність між рівнем набраного балу та чинником статі. При спробі побудови інтервалу можливих значень різниці 
середніх балів жінок та чоловіків, було отримано проміжок у $12.07\pm 0.2$ пункти, що свідчить про значне 
перевищення оцінок жінок у порівнянні з оцінками чоловіків. Крім того, значення вказаного довірчого інтервалу 
рівня довіри $\gamma=0.95$ додатково підтверджує хиб\-ність гіпотези однорідності статистичних даних результатів 
з української мови для чоловіків та жінок. 

У випадку аналізу балів з математики та англійської мови вже на першому кроці виявлено відсутність впливу статі 
на рівень отриманого балу. До того ж довірчі інтервали різниці середніх оцінок різних статей мають невеликі 
значення: $0.96\pm 0.3$ для математики та $1.89\pm 0.4$ для англійської мови. 

Отже, як результат статистичного аналізу можна стверджувати, що лише з української мови наявна статистична 
значущість відмінності результатів ЗНО в залежності від статі. Водночас на набрані бали з математики чинник 
статі не має значного впливу.

\end{document}