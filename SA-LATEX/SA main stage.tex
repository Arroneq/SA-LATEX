\documentclass[a4paper,14pt]{extarticle} % the default "article" class is limited to 12pt, but you can
                                         % go up to 14, 17 or 20 points if you use the "extarticle" class
\usepackage{cmap} % make LaTeX PDF output copy-and-pasteable
\usepackage[T2A]{fontenc}
\usepackage[utf8]{inputenc}
\usepackage[english,ukrainian]{babel}

\usepackage{amssymb, amsfonts, mathtools, amsmath, enumerate}
\usepackage{indentfirst} % set an additional space before a paragraph at the begining of a new section
\usepackage{setspace}
\usepackage{textcomp}

\usepackage{dsfont} % indicator symbol
\usepackage{leftidx} % this package enables left subscripts and superscripts in math mode

\usepackage{import} % for adding a file by path https://tex.stackexchange.com/questions/246/when-should-i-use-input-vs-include

\usepackage{geometry} 
\geometry{left=1.25cm}
\geometry{right=1.25cm}
\geometry{top=1cm}
\geometry{bottom=2cm}

\setlength{\arrayrulewidth}{0.3mm} % this sets the thickness of the borders of the table
\setlength{\tabcolsep}{12pt} % the space between the text and the left/right border of its containing cell is set to 18pt with this command
\renewcommand{\arraystretch}{1.5} % the height of each row is set to 1.5 relative to its default height

\usepackage[table,xcdraw,dvipsnames]{xcolor}
\usepackage{color}
% 1) tutorial about xcolor:  https://www.overleaf.com/learn/latex/Using_colours_in_LaTeX
% 2) huge tutorial about xcolor: https://latex-tutorial.com/color-latex/ 
% 3) RGB calculator: https://www.w3schools.com/colors/colors_rgb.asp

\usepackage{hyperref}
\definecolor{linkcolor}{HTML}{0000FF}
\definecolor{urlcolor}{HTML}{0000FF} 
\hypersetup{pdfstartview=FitH, unicode=true, linkcolor=linkcolor, urlcolor=urlcolor, colorlinks=true}

\usepackage{graphicx}
\usepackage{wrapfig}
\usepackage{float}

\parskip=1mm % space between paragraphs

% enumerating equations according to the section number 
% plus resetting each numeration inside each section
\numberwithin{equation}{section}

% numbering only sections in the table of contents (the "1" nesting level)
% thus numbering equations only according to the section number
\setcounter{secnumdepth}{1}

\usepackage{listingsutf8} % origin: \usepackage{listings}

\lstset{
    frame=single,
    language=Python,
    aboveskip=3mm,
    belowskip=3mm,
    columns=flexible,
    basicstyle=\small\ttfamily,
    numbers=left,
    numberstyle=\tiny\color{gray},
    commentstyle=\color{OliveGreen},
    stringstyle=\color{Mahogany},
    morestring=[b]''',
    showstringspaces=false,
    keywordstyle=\bfseries\color{blue},
    emph={[1]import, as, for, while, return}, emphstyle={[1]\bfseries\color{Mulberry}},
    emph={[2]range}, emphstyle={[2]\bfseries\color{brown}},
    breaklines=true,
    breakatwhitespace=true,
    tabsize=4,
    extendedchars=false, % to use ukrainian text in a code
    inputencoding=utf8 % to use ukrainian text in a code
}

\begin{document}

\import{Title/}{title}

\tableofcontents

\newpage
\import{LaTeX conversion stage/}{conversion_stage}

\newpage
\import{LaTeX UKR/}{ZNO_ukr}

\newpage
\import{LaTeX MATH/}{ZNO_math}

\newpage
\import{LaTeX ENG/}{ZNO_eng}

\newpage
\section*{Загальний висновок}
\addcontentsline{toc}{section}{Загальний висновок}

Метою статистичного аналізу було з'ясувати, чи є значущою відмінність між балами ЗНО для чоловіків 
та жінок. При цьому розглядалися результати трьох різних предметів: української мови, математики та 
англійської мови.

\subsection*{Українська мова}

Шляхом побудови довірчого інтервалу рівня довіри $\gamma=0.95$ було виявлено, що можливі значення різниці середніх балів 
жінок та чоловіків лежать у проміжку в $12.08\pm 0.2$ пункти, що свідчить про значне перевищення середніх
оцінок жінок у порівнянні з оцінками чоловіків. При цьому статистична похибка обчислень складає лише $0.21$ 
пункти. 

В свою чергу при заданій ймовірності помилки 1-го роду $\alpha=0.01$ гіпотеза однорідності даних відхиляється, 
що вказує на неоднаковість розподiлу балiв за трьома обраними категоріями оцінок \eqref{formula: marks level} 
в залежностi вiд статi.

Наостанок зауважимо, що для знормованих вибірок величина дисперсії в середньому лише у п'яти 
елементів вибірки зі ста виходить за межі проміжку від $1.32$ до $1.68$ балів.

\subsection*{Математика та англійська мова}

У випадку аналізу балів з математики та англійської мови за аналогічного рівня довіри встановлено дещо інші 
довірчі інтервали для різниці середніх оцінок різних статей: $0.94\pm 0.3$ для математики та $1.94\pm 0.42$ 
для англійської мови. Статистичні похибки обчислень при цьому складають лише $0.3$ та $0.42$ пункти відповідно. 

Для обох предметів гіпотеза однорідності даних приймається, а отже, оскільки поділ на різні категорії оцінок 
\eqref{formula: marks level} базувався на прохідних балах факультетів КПІ, то в такому разі університет може 
очікувати, що частки новоприбулих хлопців та дівчат із високими, помірними й низькими балами будуть приблизно 
однаковими.

Наприкінці зазначимо, що для знормованих вибірок величина дисперсії в середньому лише у п'яти елементів 
вибірки зі ста виходить за межі проміжку $(1.39,\ 1.94)$ балів у випадку математики й $(2.22,\ 3.20)$ балів у 
випадку англійської мови.

\subsection*{Результат}

Отже, як результат статистичного аналізу можна стверджувати, що для усіх трьох предметів відмінність середніх 
результатів ЗНО в залежності від статі є статистично значущою, причому при порівнянні середніх оцінок жінки 
та чоловіка оцінка у жінки виявиться вищою за оцінку у чоловіка для кожного з трьох предметів (українська мова, 
математика та англійська мова).

\end{document}